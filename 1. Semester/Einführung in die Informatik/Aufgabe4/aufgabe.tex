\documentclass[a4paper,11pt,titlepage]{article}

\usepackage{ucs}
% per input encoding kann man Umlaute direkt einsetzten, aber  dann ist man von Font des jeweiligen Rechners abh"angig. Daher mag ich es nicht!
%\usepackage[utf8x]{inputenc}
\usepackage[german,ngerman]{babel}
\usepackage{fontenc}
\usepackage[pdftex]{graphicx}
\usepackage{listings}
%\usepackage{latexsym}

\usepackage[pdftex]{hyperref}

\begin{document}

% hier aktuelle Uebungsnummer einfuegen
\title{Einf\"uhrung in die Informatik\\
Ausarbeitung \"Ubung 4}

% Namen der Bearbeiter einfuegen

\author{Julian Bertol}

% aktuelles Datum einfuegen

\date{\today}

\maketitle{\thispagestyle{plain}}

\section{Aufgabe 1}

\subsection{Problem}
installieren von apache und Website mir pers\"onlichen Daten erstellen.
\subsection{L\"osungskonzept}
Durch das internet herausfinden wie ma apache installiert. HTML skills schon vorhanden.
\subsection{konkrete L\"osung}
Apache ist eine Open-Source-Webserver-Software f\"ur. \newline
Diese Sprache wird auf vielen Betriebssystemen unterst\"utzt. \newline
\textbf{Installieren von Apache}
\begin{itemize}
  \item sudo apt install apache2 (installiert Apache)
\end{itemize}
\textbf{Erstellen eines Unterordners}
\begin{itemize}
  \item cd /var/www/html (In den Pfad navigieren, in dem die Seite gespeichert wird)
  \item sudo mkdir "Unterordner" (Erstellen eines Ordners Names "daten"
  \item cd "Unterordner"         (In den Ordner Daten navigieren)
  \item nano index.html  (Erstellen der haupt HTML Datei der Seite)
\end{itemize}
In unserem Fall wird die Hauptseite bearbeitet. \newline
\textbf{Bearbeiten der Hauptseite}
\begin{itemize}
  \item cd /var/www/html (In den Pfad navigieren, in dem die Seite gespeichert wird)
  \item sudo chown julian:julian index.html (gibt rechte, das man die Datein bearbeiten darf)
  \item index.html berabeiten.
\end{itemize}

\begin{lstlisting}[language=HTML]
<!DOCTYPE html>
<html lang="en">
<head>
  <meta charset="UTF-8">
  <meta name="viewport" content="width=device-width, initial-scale=1.0">
  <title>Persönliche Daten</title>
  <style> /*Optische Gestaltung*/
    *{
      background-color: #1a2f42;
      color: aliceblue;
    }
    .inline{
      display: inline;
    }
    .text_rechts{
      display: flex;
      justify-content: space-between;
      line-height: 0;
    }
    .bild{
      position: absolute;
      top: 4px;
      right: 4px;
    }
    #funktion{
      text-align: center;
    }
  </style>
</head>
<br><br><br><br><br> <!--Abstand zwischen Bild und <hr>-->
<body>
<h1 class="inline">Julian Bertol</h1>
<img class="bild" src="https://encrypted-tbn0.gstatic.com/images?q=tbn:ANd9GcSUJWPiDRfILyD7anfbnIkdpwawXmn-PpbuOD6fSqzRcQ&s" alt="Foto">
<hr>
<!--Datein einfügen-->
<h2>Geburtsdaten</h2>
<div class="text_rechts">
<h3 class="inline">Geburtsort: </h3> <p class="inline text_rechts">Berlin</p>
</div>
<div class="text_rechts">
<h3 class="inline">Geburtsdatum: </h3> <p class="inline text_rechts">11.11.2011</p>
</div>
<hr>
<h2>Persönliche Daten</h2>
<div class="text_rechts">
  <h3 class="inline">Telefon:</h3> <p class="inline text_rechts">0123 4348211</p>
</div>
<div class="text_rechts">
  <h3 class="inline">Email: </h3> <p class="inline text_rechts">Erfunden@mail.de</p>
</div>
<div class="text_rechts">
  <h3 class="inline">PLZ: </h3> <p class="inline text_rechts">72942</p>
</div>
<div class="text_rechts">
<h3 class="inline">Stadt:</h3> <p class="inline text_rechts">Berlin</p>
</div>
<div class="text_rechts">
  <h3 class="inline">Straße: </h3> <p class="inline text_rechts">Bierstraße 10</p>
</div>
<hr>
<h2>Schulischer Werdegang</h2>
<div class="text_rechts">
  <h3 class="inline">von 2010 bis 2014</h3> <p class="inline text_rechts">Grundschule</p>
</div>
<div class="text_rechts">
  <h3 class="inline">von 2014 bis 2021</h3> <p class="inline text_rechts">Gymnasium</p>
</div>
<div class="text_rechts">
  <h3 class="inline">Seit 2021</h3> <p class="inline ">student AIN HFU</p>
</div>
<br><br><br>
<div id="funktion">
  <h2>Bild der Funktion: f(x) = 10 * (sin(x))²</h2>
  <img src="funktion.png" alt="Funktion">
</div>
</body>
</html>
\end{lstlisting}

\subsection{Tests}
um installation zu \"uberpr\"ufen im Browser die IP des rechners oder "Localhost" eingeben.
\section{2 Python und Matplotlib}
\subsection{2.1 Python installieren}
\begin{itemize}
  \item Installation von Pycharm. Hier ist das Paket Python automatisch dabei.
  \item Installation von Matplotlib (pip install matplotlib numpy)
  \item alternativ kann man auch im terminal mit dem befehl: "sudo apt install python3"
\end{itemize}
\subsection{2.2 Was ist Matplotlib}
\begin{itemize}
  \item Eine Bilbiothek mit der man Mathematische funktionen darstellen kann
\end{itemize}

\section{Aufgabe 3}
% hier dann die eigene Bearbeitung einfuegen
....

% ---------
\section{Resumee zur dieser \"Ubungsaufgabe}
Dauer f\"ur 
\begin{itemize}
	\item Durchf\"uhrung
	\item Dokumentation
\end{itemize}
Welche großen Probleme waren zu l\"osen?

\begin{thebibliography}{99}
% per \bibitem werden die Eintraege hier eingefuegt und per \cite{identifier}
% im Text oben referenziert
\bibitem{lkurz} 
	Walter Schmidt, J\"org Knappen, Hubert Partl, Irene Hyn: 
\LaTeXe-Kurzbeschreibung, 	Version 2.3, 2003
	

\end{thebibliography}
\end{document}
