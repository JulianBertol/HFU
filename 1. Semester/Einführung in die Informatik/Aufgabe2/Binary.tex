\documentclass[a4paper,11pt,titlepage]{article}

\usepackage{ucs}
% per input encoding kann man Umlaute direkt einsetzten, aber  dann ist man von Font des jeweiligen Rechners abh"angig. Daher mag ich es nicht!
%\usepackage[utf8x]{inputenc}
\usepackage[german,ngerman]{babel}
\usepackage{fontenc}
\usepackage[pdftex]{graphicx}
%\usepackage{latexsym}

\usepackage[pdftex]{hyperref}

\begin{document}

% hier aktuelle Uebungsnummer einfuegen
\title{Einf\"uhrung in die Informatik\\
Ausarbeitung \"Ubung 2}

% Namen der Bearbeiter einfuegen

\author{Julian Bertol}

% aktuelles Datum einfuegen

\date{\today}

\maketitle{\thispagestyle{plain}}

\section{Aufgabe 1}
\textbf{1 Umrechnung zwischen Zahlensystemen} \newline
Uben Sie die Umrechnung von verschiedenen Ausgangs-Zahlensystemen in verschiedene Basissysteme:\newline
Folgende Zahlen sollen umgerechnet werden:\newline
19210,0C16,7648,011111102\newline
Berechnen Sie fur jede Zahl die Darstellung in den anderen hier verwendeten Zahlensystemen.\newline
Bestimmen Sie fur die nachfolgenden aufeinander aufbauenden Festlegungen jeweils den unteren und oberen darstellbaren Wert im Dezimalsystem und im Hexadezimalsystem.

\subsection{Problem}
Folgende Zahlen müssen umgerechnet werden:\newline
19210,0C16,7648,011111102 \newline
\subsection{L"osungskonzept}
Verstehen wie die umrechnunng funktioniert mit hilfe von Google \newline
\pagebreak

\subsection{konkrete L"osung}

$192_10$ ist eine Dezimalzahl, da es sich um das 10er Zahlensystem handelt.\newline
Umrechnung in Bin\"ar \newline
192 / 2 = 96 Rest 0 \newline
 96 / 2 = 48 Rest 0 \newline
 48 / 2 = 24 Rest 0 \newline
 24 / 2 = 12 Rest 0 \newline
 12 / 2 =  6 Rest 0 \newline
  6 / 2 =  3 Rest 0 \newline
  3 / 2 =  1 Rest 1 \newline
  1 / 2 =  0 Rest 1 \newline
192 in Bin\"ar ist = 11000000 \newline

192 in Hexadezimal \newline
192 / 16 = 12 Rest 0 \newline
 12 / 16 =  0 Rest 12 \newline
192 in Hexadezimal = C0 \newline

192 in Oktalzahlensystem \newline
192 / 8 = 24 Rest 0 \newline
 24 / 8 =  3 Rest 0 \newline
  3 / 8 =  0 Rest 3 \newline
192 in Oktal = 300 \newline



$oC_16$ in Dezimal: \newline
= 12 * 16⁰ + 0 * 16¹ \newline
= 12 \newline

$oC_16$ in Bin\"ar \newline
12 / 2 = 6 Rest 0 \newline
 6 / 2 = 3 Rest 0 \newline
 3 / 2 = 1 Rest 1 \newline
 1 / 2 = 0 Rest 1 \newline
$oC_16$ in Bin\"ar = 1100

$oC_16$ in Oktal \newline
12 / 8 = 1 Rest 4\newline
 1 / 8 = 0 Rest 1\newline
$oC_16$ in Oktal = 14 \newline



\pagebreak

$764_8$ Oktal in Dezimal \newline
= 4 * 8⁰ + 6 * 8¹ + 7 * 8² \newline
= 500\newline

$764_8$ in Hexadezimal \newline
500 / 16 = 31 Rest 4 \newline
 31 / 16 =  1 Rest 15 \newline
  1 / 16 =  0 Rest 1 \newline
$764_8$ in Hexa = 1F4 \newline

$764_8$ in Bin\"ar \newline
500 / 2 = 250 Rest 0 \newline
250 / 2 = 125 Rest 0 \newline
125 / 2 =  62 Rest 1 \newline
 62 / 2 =  31 Rest 0 \newline
 31 / 2 =  15 Rest 1 \newline
 15 / 2 =   7 Rest 1 \newline
  7 / 2 =   3 Rest 1 \newline
  3 / 2 =   1 Rest 1 \newline
  1 / 2 =   0 Rest 1 \newline
 $764_8$ in Bin\"ar = 111110100





$01111110_2$ von Bin\"ar in Dezimal \newline
= 2¹+2²+2³+2⁴+2⁵+2⁶ \newline
= 126 \newline

$01111110_2$ in Hexadezimal \newline
126 / 16 = 7 Rest 14 \newline
  7 / 16 = 0 Rest  7 \newline
$01111110_2$ in Hexa = 7E

$01111110_2$ in Oktal \newline
126 / 8 = 15 Rest 6 \newline
 15 / 8 =  1 Rest 7 \newline
  1 / 8 =  0 Rest 1 \newline
$01111110_2$ in Oktal = 176


 

HexadezimalSystem erklärt: \newline
Zahlen von 0-9 wie auch im normalen DezimalSystem. \newline
Buchstaben von A bis F\newline
A = 10 \newline
B = 11 \newline
C = 12 \newline
D = 13 \newline
E = 14 \newline
F = 15 \newline

Oktalsystem erklärt: \newline
Oktal bedeutet 8. \newline
Man nimmt also jeden koeffizienten und multipliziert ihn mit 8 und eine Hochzahl \newline
Dabei beginnt man von rechts nach links und erhöt die Zahl hochzahl immer um 1. Man beginnt mit 0. \newline
Am ende werden alle Zahlen addiert.\newline

Bin\"arsystem erklärt: \newline
Man hat eine Zahlenfolge von 0 und 1. Man beginnt von rechts nach links zu lesen. Ist die Zahl 0, \newline
kann man diese \"uberspringen. Ist die Zahl 1 rechnet man mit 2 und eine Hochzahl. Dabei wird die hochzahl von rechts nach links \newline
immer um 1 erh\"ot ergal um die Bin\"ar Zahl 1 oder 0 ist. Am ende addiert man alle Zahlen zusammen.





\textbf{Beispiel IPV4 Adressen} \newline
\begin{itemize}
	\item Das H\"ochstwerige Bit muss 0 sein: Die Wertdastellung geht von 0 bis 127:\newline
	      011111111 = 127\newline
	\item Jetzt muss das H\"ochstwertige Bit immer 1 sein, das Zweith\"ochste Bit muss 0 sein:\newline
	      Die Wertedarstellung get von: 128 bis 119 \newline
	      Niedrigste: 10000000 = 128 \newline
	      H\"ochste:  10111111 = 191 \newline
	\item jetzt m\"ussen das h\"ochste und das zweith\"ochste Bit 1 gesetzt sein, das dritth\"ochste \newline
	      Bit muss 0 sein: Die Wertedarstellung geht von 0 bis 10 \newline
	      Niedrigste: 11000000 = 192 \newline
	      H\"ochste:  11011111 = 232 \newline
\end{itemize}

\pagebreak

\subsection{Tests}
Nach einen Umrechner im Internet suchen und die ergebnisse vergleichen. \newline

\section{Aufgabe 2}
% hier dann die eigene Bearbeitung einfuegen
\subsection{Problem}
\begin {itemize}
\item zuerst muss man die Gr\"o\"ste 4 Bit Zahl in Dezimal umrechnen \newline
\item Dann muss eine Tabelle ausgefüllt werden indem man in verschiedenen Systemen umrechnet \newline
\end {itemize}

\subsection{L\"osungskonzept}
Mit den vorher gelerneten eigenschaften die L\"osungen berechnen \newline

\pagebreak
\subsection{L\"osung}
Die gr\"o\"ste 4 bit zahl im Bin\"arsystem ist: \newline
1111 = 15 \newline
Die kommastellen werden mit $1*2^{-n}$ berechnet. Dabei ist n die Bitzahl \newline
Berechnung der Kommastelle: \newline
$(1*2^{-1}) + (1*2^{-2}) + (1*2^{-3})+ (1*2^{-4})$ \newline
Das ergibt dann $0.9375$ \newline
Somit ist die gr\"o\"ste darstellbare Zahl im Bin\"arsystem mit 4 bit 15.9375 \newline


Die gr\"o\"ste 4 bit Zahl im Hexadezimalsystem ist: \newline
FFFF = 65535 \newline
Auch hier werden die Nachkomma zahl mit $15*16^{-n}$ berechnet \newline
$(15 * 16^{-1}) + (15 * 16^{-2}) + (15 * 16^{-3}) + (15 * 16^{-3})$ \newline
somit ist das ergebniss : $.9999847412109375$ \newline


Die gr\"o\"ste 4 bit Zahl im Hexadezimalsystem ist: \newline
$7777 = 4095$
Die Zahlen nach dem Komma m"ussen hier dann als $7 * 8^{-n}$ geschrieben werden. Somit $(7 * 8^{-1}) + (7 * 8^{-2}) + (7 * 8^{-3}) + (7 * 8^{-4})$ was ca.999755859375 ergibt. \newline
Somit ist die Gr\"o\"ste Zahl 4095.999755859375 \newline

\begin{table}
\begin{tabular}{lll}
Dualsystem        & Oktalsystem & Hexadezimalsystem \\
101101.101        & 55.5        & 2D.A              \\
10101011.11001101 & 253.632     & AB.CD             \\
\end{tabular}
\end{table}
    

\pagebreak

\section{Aufgabe 3}
textbf{Bin\"are Addition}
125 + 199 \newline
In Bin\"ar umwandeln: \newline
01111101 \newline
11000111 \newline
=101000100 \newline
in Dezimal 324 \newline

27+30 \newline
11011 \newline
11110 \newline
=111001 \newline
in Dezimal: 57 \newline

115 + 21 \newline
1110011 \newline
0010101\newline
=10001000 \newline
in Dezimal 136 \newline

\pagebreak

\textbf{Bin\"are Subtraktion}
55 - 120\newline
00110111\newline
01111000- \newline
Zuerst muss das unter dem Bruch in ein Zweierkomplement umgerechnet werden \newline
Dabei wird die 1 zur 0 und die 0 zur 1. Am Ende wird noch +1 addiert \newline
10000111\newline
00000001+ \newline
= 10001000 \newline
Dieses Ergebniss wird dann vom ersten im Urspr\"unglichen Bruch abgezogen \newline
00110111\newline
10000111- \newline
Das ergibt dann 10111111 \newline



42 - 12 \newline
Umrechnen in Binär:\newline
00101010 \newline
00001100- \newline
=00011110 \newline
Hier normale Subtraktion verwendet. \newline
Wichtige Regeln: \newline
1-1 = 0 \newline
0-1 = 1 Rest 1 \newline
1-0 = 1 \newline
0-0 = 0 \newline

18 - 105 \newline
Umrechnen in Binär: \newline
00010010 \newline
01101001- \newline
=10101001 \newline


% ---------
\section{Resumee zur dieser "Ubungsaufgabe}
Insgesamt habe ich c.a 4 Stunden mit dieser Aufgabe verwendet.
\begin{itemize}
	\item Durchf"uhrung
	\item Dokumentation
\end{itemize}
Welche gro"sen Probleme waren zu l"osen?
Ich musste mir selber Wissen \"uber die verschiedenn Systeme aneignen. Das war nicht immer einfach, da dass Internet dort nicht immer eine Hilfe sein konnte.

\begin{thebibliography}{99}
% per \bibitem werden die Eintraege hier eingefuegt und per \cite{identifier}
% im Text oben referenziert
\bibitem{lkurz} 
	Walter Schmidt, J"org Knappen, Hubert Partl, Irene Hyn: 
\LaTeXe-Kurzbeschreibung, 	Version 2.3, 2003
	

\end{thebibliography}
\end{document}
