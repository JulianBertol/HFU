\documentclass[a4paper,11pt,titlepage]{article}

\usepackage{ucs}
\usepackage{listings}
% per input encoding kann man Umlaute direkt einsetzten, aber  dann ist man von Font des jeweiligen Rechners abh"angig. Daher mag ich es nicht!
%\usepackage[utf8x]{inputenc}
\usepackage[german,ngerman]{babel}
\usepackage{fontenc}
\usepackage[pdftex]{graphicx}
%\usepackage{latexsym}

\usepackage[pdftex]{hyperref}

\begin{document}

% hier aktuelle Uebungsnummer einfuegen
\title{Einf\"uhrung in die Informatik\\
Ausarbeitung \"Ubung 2}

% Namen der Bearbeiter einfuegen

\author{Julian Bertol}

% aktuelles Datum einfuegen

\date{\today}

\maketitle{\thispagestyle{plain}}

\section{compile and debug}
\subsection{Problem}
Ich muss einen euklerischen Algoritmus erstellen. Diesen dann mit gcc oder g++ compelieren.\newline
Danach muss ich erkl\"aren was gcc, g++ und gdb ist. \newline
\subsection{L\"osungskonzept}
Ich erstelle das Prgramm und teste dann was gcc, g++ und gdb machen. Wenn ich nicht weiter weis benutze ich das internet. \newline
\subsection{konkrete L\"osung}
euklerischer Algortihmus: \newline
\begin{lstlisting}[language=c++]
#include <stdio.h>
int main() {
    int number1;
    int number2;
    int ergebnis;
    int ergebnis_rest;
    printf("Geben Sie die erste Zahl ein: \n");
    scanf("%i", &number1); 
    printf("Geben Sie die zweite Zahl ein: \n"); 
    scanf("%i", &number2);
    do { 
        ergebnis = number1 / number2;
        ergebnis_rest = number1 % number2;
        number1 = number2;
        number2 = ergebnis_rest;
    } while (ergebnis_rest > 0);
    printf("Das Ergebnis lautet: %i\n", number1);
    return 0;
} 
\end{lstlisting}
\pagebreak
Zum compelieren müssen wir mit dem command cd in das Verzeichniss gehen wo die Datei liegt. Danach compelieren wird die c++ datei entweder mit dem Befehl: \newline
g++ "dateiname" -o "neuer Dateiname" \newline
oder \newline
gcc "dateiname" -o "neuer Dateiname" \newline
mit dem befehl chmod +x "dateiname" \newline
muss man die Zugrifssrechte auf das asuführen der Datei bearbeiten. \newline
Nun kann man mit dem Befehl ; \newline
./"dateiname"\newline
die compeliete Datei ausführen. \newline

\textbf{gdb} ist ein Debugger. Man kann den Debugger mit dem Befehl: \newline
gdb "dateiname" aufrufen. \newline
\subsection{Tests}
Ausführen des programmes und nachrechnen. \newline

\section{Aufgabe 2 Build}
\subsection{2.1 make}
Make ist ein build management tool. Ein Build-Management-Tool ist ein Softwarewerkzeug,
das den Prozess der Kompilierung, des Testens und der Bereitstellung von Quellcode automatisiert,
um eine ausführbare Anwendung oder ein Softwarepaket zu erstellen. Dieser Prozess wird als "Build"
bezeichnet, und das Build-Management ist entscheidend für die Entwicklung von Softwareprojekten.
Einige der Hauptaufgaben eines Build-Management-Tools umfassen \newline
\textbf{erstellen einer makefile} \newline
Zuerst erstellt man eine .make datei. In dieser werden dann Infortmationen zu make gespeichert. \newline
\pagebreak
\begin{lstlisting}[language=make]
# Variable für Compiler-Flags. Das W steht für Warnungen aktivieren
# und das g für Debug-Informationen
Coption = -W -g

# Regel für das Standardziel "all"
all: euklerisch

# Regel für das Ziel "euklerisch"
euklerisch: euklerisch.cpp
	g++ euklerisch.cpp $(Coption) -o euklerisch

# Regel für das Ziel "ex" (Ausführen)
ex: euklerisch
	./euklerisch

# Regel für das Ziel "db" (Debugging)
db: euklerisch
	gdb euklerisch

# Regel für das Ziel "rm" (Löschen)
rm: euklerisch    
	rm euklerisch
	rm euklerisch.o
\end{lstlisting}


\subsection{2.2 cmake}
Cmake ist an sich das selbe wie make, nur dass cmake auf allen betriebsystemen anwendbar ist und make nur auf Unix systemen. \newline
\begin{lstlisting}[language=make]
cmake_minimum_required(VERSION 3.22.1) #Version angeben ist hielfreich 
#fuer komplexe Projekte, die spezifische Funktionen oder Befehle
#beim Build besitzen
project (euklerisch) #Name des Projekts
add_executable(euklerisch euklerisch.cpp) #Erstellt passende
#build-Dateien f"ur das Projekt euklerisch.cpp
\end{lstlisting}
\begin{itemize}
  \item \textbf{Worin unterscheidet sich das Ziel von make und cmake? Beschreiben Sie in eigenen Worten die Aufgabe von make.} \newline
        Make ist eine automatisierende anwendung die den compiler automatisiert. Außerdem kann man sein Programm debuggen ohne IDEA. \newline
        Der unterschied zwischen make und cmake besteht darin, dass cmake auf mehreren Plattformen ausführbar ist und make nur auf Unix.
  \item \textbf{Was sollte das PHONY-Target clean machen? }\newline
        .phony sorgt daf\"ur, dass der Befehl clean ausgef\"uhrt wird und make nicht nach einer Datei namens clean sucht und diese versucht ausf\"uhrbar zu machen.
        Das Phony target clean sollte die Makefile2-Datei im CMakeFiles Odrner bereinigen.
  \item \textbf{Wie wurde die Funktionserfullung gepr\"uft?}\newline
        Die Funktionserf"ulung wurde \"uberpr\"uft, indem die Makefile Datei mit dem Befehl make ausgef\"uhrt wurde.
        Es entsteht eine ausf\"uhrbares Programm, welches man mit dem Terminalbefehl  ./"`Dateiname"' ausf\"uhren kann.
        Durch einsetzen von Testwerten kann man erkennen, dass das Prgramm das macht, was es soll.
\end{itemize}
\section{Resumee zur dieser "Ubungsaufgabe}
Dauer f\"ur 
\begin{itemize}
	\item Durchf\"uhrung 1.5 Stunden
	\item Dokumentation 0.5 Stunden
\end{itemize}
Welche großen Probleme waren zu l\"osen?
Die make-file war mein gr\"oßtes Problem, da ich mich nicht mit diesem System auskannte.
\end{document}
